Growth curve data on an orthdontic measurement


Description

The Orthodont data frame has 108 rows and 4 columns of the change in an orthdontic measurement 
over time for several young subjects.


Details

Investigators at the University of North Carolina Dental School followed the growth of 27 children (16 males, 11 females) from age 8 until age 14. Every two years they measured the distance between the pituitary and the pterygomaxillary fissure, two points that are easily identified on x-ray exposures of the side of the head.
%============================================================================%
Format

This data frame contains the following columns:

\texttt{distance}
a numeric vector of distances from the pituitary to the pterygomaxillary fissure (mm). These distances are measured on x-ray images of the skull.

\texttt{age}
a numeric vector of ages of the subject (yr).

\texttt{Subject}
an ordered factor indicating the subject on which the measurement was made. The levels are labelled M01 to M16 for the males and F01 to F13 for the females. The ordering is by increasing average distance within sex.

\texttt{Sex}
a factor with levels Male and Female

%============================================================================%
\end{document}
