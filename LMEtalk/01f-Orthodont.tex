Growth curve data on an orthdontic measurement


Description

The Orthodont data frame has 108 rows and 4 columns of the change in an orthdontic measurement 
over time for several young subjects.


Details

Investigators at the University of North Carolina Dental School followed the growth of 27 children (16 males, 11 females) from age 8 until age 14. Every two years they measured the distance between the pituitary and the pterygomaxillary fissure, two points that are easily identified on x-ray exposures of the side of the head.
%============================================================================%
Format

This data frame contains the following columns:

\texttt{distance}
a numeric vector of distances from the pituitary to the pterygomaxillary fissure (mm). These distances are measured on x-ray images of the skull.

\texttt{age}
a numeric vector of ages of the subject (yr).

\texttt{Subject}
an ordered factor indicating the subject on which the measurement was made. The levels are labelled M01 to M16 for the males and F01 to F13 for the females. The ordering is by increasing average distance within sex.

\texttt{Sex}
a factor with levels Male and Female
\end{frame}
%=========================================================================================================%
\begin{frame}[fragile]

\begin{verbatim}
> library(nlme)
> fm1 <- lme(distance~age,random=~1|Subject,data=Orthodont)
> coef(fm1)
    (Intercept)       age
M16    15.84314 0.6601852
M05    15.84314 0.6601852
M02    16.17959 0.6601852
M11    16.40389 0.6601852
M07    16.51604 0.6601852
M08    16.62819 0.6601852
M03    16.96464 0.6601852
\end{verbatim}
\end{frame}
%=========================================================================================================%
\begin{frame}
use fixef() to get just the fixed effect coefficients
use ranef() to get just the random effects (i.e. deviations of each individual from the fixed coefficients
the Orthodont example in lme actually uses a random-slope(+intercept) model; here I have fitted a random-intercept model, so the estimated slope (age parameter) is the same for every individual
it looks like individuals are sorted in increasing order of estimated random effect
%============================================================================%

I have two factors in the linear mixed model. Factor A is treated as fixed effect, factor B is treated as random effect and nested into factor A. Can anyone tell me how to do this using nlme R package?

I know that lme( response~ factorA, random=~1|factorA/factorB) is one way to model. however, this function treat factor A as random effect.

This depends on how your variables are coded. You might have distinct names for the variables in factorB, like this; then just having factorB as a random effect, is sufficient.

factorA  factorB
bob      bob1
bob      bob2
bob      bob3
jane     jane1
jane     jane2
jane     jane3

lme(response ~ factorA, random=~1|factorB)
But you might have the same coding for the variables in factorB for each level of factorA, like this; then just having factorB as a random effect is not correct; you instead need the random effect to be the interaction between them, I think code using : would work, but it might be more readable to make a new variable.

factorA  factorB
bob      rep1
bob      rep2
bob      rep3
jane     rep1
jane     rep2
jane     rep3

%==============================================================================================%
lme(response ~ factorA, random=~1|factorA:factorB)

dat$factorAB <- with(dat, factor(paste(factorA, factorB), sep="."))
lme(response ~ factorA, random=~1|factorAB)
%==============================================================================================%
\end{document}
