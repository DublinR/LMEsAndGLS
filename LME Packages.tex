LME Packages

nlme is the most mature one and comes by default with any R installation. 
In addition to fitting hierarchical generalized linear mixed models it also allows 
fitting non-linear ones. 

Its main advantages are, in my humble opinion, the ability to fit fairly complex hierarchical 
models using linear or non-linear approaches, a good  variety of variance and correlation 
structures, and access to several distributions and link functions for generalized models. 

In my opinion, its main drawbacks are 
i- fitting cross-classified random factors is a pain, 
ii- it can be slow and may struggle with lots of data, 
iii- it does not deal with pedigrees by default and 
iv- it does not deal with multivariate data.

%----------------------------------------------------------------------------------%
lme4 is a project led by Douglas Bates (one of the co-authors of nlme), 
looking at modernizing the code and making room for trying new ideas. 

On the positive side, it seems to be a bit faster than nlme and it deals a lot
better with cross-classified random factors. 

Drawbacks: similar to nlme’s, but dropping point 
i- and adding that it doesn’t deal with covariance and correlation structures yet. 

It is possible to fit pedigrees using the mmpedigree package, but I find the combination a bit flimsy.
